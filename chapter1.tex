\chapter{บทนำ}
\label{chapter:introduction}

\section{ที่มาและความสำคัญ}
ในด้านการเรียนรู้ของเครื่องจักร (Machine Learning) ความไม่สมดุลกันของข้อมูล หมายถึง การที่จำนวนตัวอย่างของแต่ละกลุ่มข้อมูลมีจำนวนไม่เท่ากัน 
ซึ่งความไม่สมดุลกันของข้อมูลนี้ถูกนิยามให้เป็นปัญหาในการจัดกลุ่มข้อมูล (Classification) สาเหตุที่ความไม่สมดุลกันของข้อมูลเป็นปัญหา คือ อัลกอริทึมการจัดกลุ่ม 
(Classification Algorithm) จะทำงานได้อย่างมีประสิทธิภาพก็ต่อเมื่อจำนวนตัวอย่างของแต่ละกลุ่มข้อมูลมีจำนวนที่เท่าหรือใกล้เคียงกัน 
เมื่อมีความไม่สมดุลกันของข้อมูลจะทำให้การทำงานของอัลกอริทึมมีประสิทธิภาพด้อยลง ซึ่งอาจจะด้อยลงจนไม่สามารถจัดกลุ่มข้อมูลได้เลย

การจัดกลุ่มข้อมูลที่มีจำนวนตัวอย่างของแต่ละกลุ่มไม่สมดุลกันเป็นเรื่องธรรมดาอย่างมากในทางปฏิบัติ 
เนื่องจากข้อมูลที่เกิดขึ้นล้วนแต่ไม่สามารถคาดเดาได้อย่างแน่นอนว่าจำนวนตัวอย่างของแต่ละกลุ่มจะสมดุลกัน อีกทั้งข้อมูลส่วนใหญ่ยังมีลักษณะที่มีจำนวนตัวอย่างของแต่ละกลุ่มไม่สมดุลกัน 
เช่น ในระหว่างวัวอยู่ในช่วงเป็นสัด ช่วงเวลาที่วัวแสดงพฤติกรรมเป็นสัดจะมีจำนวนน้อยกว่าช่วงเวลาที่วัวไม่แสดงพฤติกรรมเป็นสัด เป็นต้น ในด้านการเรียนรู้ของเครื่องจักร 
มีความเป็นไปได้ว่าตัวจัดกลุ่มข้อมูลที่ถูกสร้างขึ้นจากชุดข้อมูลที่มีจำนวนตัวอย่างของแต่ละกลุ่มไม่สมดุลกันจะมีความลำเอียงในการจัดกลุ่ม กล่าวคือ 
มีโอกาสสูงที่ตัวจัดกลุ่มจะระบุว่าข้อมูลเป็นกลุ่มส่วนมาก (Majority Class) มากกว่าเป็นกลุ่มส่วนน้อย (Minority Class) 
ซึ่งเป็นผลทำให้การระบุข้อมูลเป็นกลุ่มส่วนน้อยมีความแม่นยำที่ต่ำกว่ามาตรฐาน ซึ่งความแม่นยำในการระบุข้อมูลเป็นแต่ละกลุ่มควรจะเท่าหรือใกล้เคียงกัน

ที่ผ่านมาได้มีการศึกษาเกี่ยวกับปัญหาการจัดกลุ่มข้อมูลในลักษณะนี้อย่างกว้างขวาง และได้แสดงให้เห็นถึงความสำคัญของปัญหาของการจัดกลุ่มข้อมูลที่มีจำนวนตัวอย่างของแต่ละกลุ่มไม่สมดุลกัน 
ซึ่งทำให้ประสิทธิภาพการจัดกลุ่มข้อมูลมีความแม่นยำที่ต่ำ ดังนั้นปัญหานี้จำเป็นต้องถูกจัดการ \cite{Buda:2017} เพื่อที่จะแก้ปัญหาการจัดกลุ่มข้อมูลที่มีจำนวนตัวอย่างของแต่ละกลุ่มไม่สมดุลกัน 
ได้มีเทคนิคเกิดขึ้นมากมาย โดยสามารถแบ่งเทคนิคการแก้ปัญหาได้ 2 ระดับ คือ (1) ระดับข้อมูล (Data-Level) 
ที่ซึ่งเป็นการแก้ปัญหาโดยการจัดการข้อมูลก่อนที่จะถูกนำไปประมวลในกระบวนการจัดกลุ่มข้อมูล โดยการสุ่มเพิ่มจำนวนตัวอย่างข้อมูล (Over-Sampling) และการสุ่มลดจำนวนตัวอย่างข้อมูล 
(Under-Sampling) เป็นเทคนิคในการแก้ปัญหาในระดับข้อมูล เทคนิคการแก้ปัญหาในระดับนี้เป็นการแก้ปัญหาแบบเบื้องต้นที่สามารถดำเนินการได้ง่าย 
อย่างไรก็ตามการสุ่มเพิ่มจำนวนตัวอย่างข้อมูลสามารถทำให้เกิดปัญหา Overfitting ตามมาได้อย่างง่ายดาย 
ในทางเดียวกันการสุ่มลดจำนวนตัวอย่างข้อมูลอาจจะเป็นการกำจัดสารสนเทศที่เป็นประโยชน์ต่อการจัดกลุ่มข้อมูลออกไป (2) ระดับตัวจัดกลุ่ม (Classifier-Level) 
ที่ซึ่งเป็นการแก้ปัญหาโดยการจัดการอัลกอริทึมการจัดกลุ่ม โดยการทำเทรสโช (Thresholding) การเรียนรู้แบบความเสียหายที่รู้สึกได้ง่าย (Cost-Sensitive Learning) 
การจัดกลุ่มข้อมูลแบบหนึ่งกลุ่ม (One-Class Classification) และการผนวกกันของหลายเทคนิค อย่างไรก็ตามเทคนิคเหล่านี้มียังไม่สามารถแก้ปัญหาได้อย่างมีประสิทธิภาพในทุก ๆ ชุดข้อมูล 
กล่าวคือ เทคนิคสามารถให้ความแม่นยำในการจัดกลุ่มข้อมูลแต่ละกลุ่มได้อย่างน่าพอใจสำหรับชุดข้อมูล A แต่ไม่สามารถทำได้อย่างมีประสิทธิภาพสำหรับชุดข้อมูล B เป็นต้น 
ดังนั้นเทคนิคใหม่ที่จะสามารถการแก้ปัญหาการจัดกลุ่มข้อมูลที่ไม่สมดุลกันได้อย่างมีประสิทธิภาพ และปรับเข้าได้กับทุกชุดข้อมูลจำเป็นต้องถูกคิดค้นขึ้น

\section{ความมุ่งหมายและวัตถุประสงค์ของการศึกษา}
\begin{enumerate}
	\item เพื่อศึกษาการจัดการปัญหาความไม่สมดุลกันของข้อมูลด้วยเทคนิคต่าง ๆ
\end{enumerate}
\section{ขอบเขตการพัฒนาโครงงาน}
\begin{enumerate}
	\item ทดลองใช้เทคนิคการจัดกลุ่มข้อมูลที่ไม่สมดุลกันแบบต่าง ๆ ในแต่ละชุดข้อมูล เพื่อค้นหาว่าเทคนิคที่ดีที่สุด
\end{enumerate}
\section{ขั้นตอนการดำเนินงาน}
\begin{enumerate}
	\item ศึกษาเกี่ยวกับนิยามของความไม่สมดุลกันของข้อมูลในด้านการจัดกลุ่มข้อมูล
	\item ศึกษารูปแบบการแก้ปัญหาการจัดกลุ่มข้อมูลที่ไม่สมดุลกันแบบต่าง ๆ
	\item ศึกษางานวิจัยที่เกี่ยวข้อง
	\item ตั้งข้อสมมติฐาน
	\item ออกแบบการทดลอง
	\item เลือกชุดข้อมูล และ Metrics ที่จะใช้ในการทดลอง
	\item ดำเนินการทำการทดลอง
	\item สรุปผลการทดลอง
\end{enumerate}
\section{ประโยชน์ที่คาดว่าจะได้รับ}
\begin{enumerate}
	\item เทคนิคในการสร้างแบบจำลองการจัดกลุ่มข้อมูลที่สามารถให้ความแม่นยำที่สูงในการระบุข้อมูลแต่กลุ่ม แม้ว่าแบบจำลองจะเรียนรู้จากชุดข้อมูลที่ไม่สมดุลกัน
\end{enumerate}

