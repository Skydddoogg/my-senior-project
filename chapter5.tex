\chapter{บทสรุป}
\label{chapter:conclusion}

ในงายวิจัยนี้ ผู้วิจัยได้ศึกษาการทำงานของฟังก์ชันสูญเสียแบบ Mean False Error (MFE) และ Focal Loss (FL) สำหรับการเรียนรู้ของแบบจำลองโครงข่ายประสาทเทียมเชิงลึกอย่างละเอียด เพื่อที่จะหาข้อได้เปรียบของแต่ละฟังก์ชัน ฟังก์ชันสูญเสียทั้งสองดังกล่าวเป็นฟังก์ชันสูญเสียที่ถูกออกแบบมาเพื่อจัดการกับปัญหาความไม่สมดุลกันของข้อมูล
จากผลการศึกษาพบว่าแต่ละฟังก์ชันสูญเสียมีวิธีการจัดการปัญหาความไม่สมดุลกันของข้อมูลที่ต่างกัน และแนวคิดของแต่ละฟังก์ชันสามารถนำมารวมกันได้ เพื่อที่จะใช้ข้อได้เปรียบของทั้งสองฟังก์ชันเพิ่มประสิทธิภาพการเรียนรู้ของโมเดล ด้วยเหตุนี้ผู้วิจัยจึงนำเสนอฟังก์ชันสูญเสียแบบใหม่ที่เรียกว่า Hybrid Loss ที่ซึ่งเป็นการผสมผสานกันระหว่าง MFE และ FL โดยการคำนวณค่าสูญเสียแต่ละตัวอย่างข้อมูลจะคำนวณด้วย FL แต่สำหรับการคำนวณค่าสูญเสียรวมจะนำวิธีการคำนวณค่าสูญเสียรวมของ MFE มาใช้ ผลการทดลองการจัดกลุ่มข้อมูลสองกลุ่มด้วยชุดข้อมูลที่หลากหลายแสดงให้เห็นว่า Hybrid Loss สามารถให้ผลการทดลองในเชิงความแม่นยำที่สูงกว่า MFE และ FL อย่างชัดเจน ซึ่งนั้นหมายความ Hybrid Loss สามารถทำงานได้อย่างดีในการจัดกลุ่มข้อมูลสองกลุ่ม ในอนาคตผู้วิจัยอาจจะทดลอง Hybrid Loss กับการจัดกลุ่มข้อมูลหลายกลุ่ม เพื่อพิสูจน์ว่าแนวคิดของ Hybrid Loss ในตอนนี้สามารถจัดการกับปัญหาความไม่สมดุลกันของข้อมูลหลายกลุ่มได้หรือไม่