\documentclass{itkmitlproject}

\usepackage{afterpage}
\usepackage{graphicx,amsmath,latexsym,amssymb,amsthm}
\usepackage{indentfirst}
\usepackage{cite}
\usepackage{placeins}
\usepackage[super,square]{natbib}
\usepackage{booktabs}
\usepackage{multirow}

\setmainfont{Angsana}[
 Path = fonts/,
 UprightFont = Angsana,
 ItalicFont = *-Italic,
 BoldFont = *-Bold,
 BoldItalicFont = *-BoldItalic,
    Extension = .ttf,
    WordSpace = 0.9,
    Scale = 1.6
]

\graphicspath{ {images/} }

\makeatletter
% \patchcmd{<cmd>}{<search>}{<replace>}{<succes>}{<failure>}
\patchcmd{\@chapter}{\addtocontents{lof}{\protect\addvspace{10\p@}}}{}{}{}% LoF
\patchcmd{\@chapter}{\addtocontents{lot}{\protect\addvspace{10\p@}}}{}{}{}% LoT
\def\bibindent{2.1em}
\let\old@biblabel\@biblabel
\def\@biblabel#1{\old@biblabel{#1}\kern\bibindent}
\let\old@bibitem\bibitem
\def\bibitem#1{\old@bibitem{#1}\leavevmode\kern-\bibindent}
\makeatother

%1. Your thesis title (THAI)
\newcommand{\ThesisTitle}{ฟังก์ชันสูญเสียสำหรับการเรียนรู้ข้อมูลที่ไม่สมดุล}
%2. Your thesis title (ENG)
\newcommand{\ThesisTitleENG}{Hybrid Loss for Learning Imbalanced Data}
%3. Author name
\newcommand{\Author}{ธนวัฒน์ หลอดแก้ว}
%4. Author name ENG
\newcommand{\AuthorENG}{Thanawat Lodkaew}
%5. Author student ID
\newcommand{\SId}{59070071}
%9. Advisor name
\newcommand{\Advisor}{รองศาสตราจารย์ ดร. กิติ์สุชาต  พสุภา}
%10. Advisor name ENG
\newcommand{\AdvisorENG}{Assoc. Prof. Dr. Kitsucart  Pasupa}
%11. ภาคเรียนที่
\newcommand{\Sem}{1}
%12. ปีการศึกษา พ.ศ.
\newcommand{\AcaY}{2562}
%13. ปีการศึกษา ค.ศ.
\newcommand{\AcaYAD}{2019}
%14. วันส่งรายงาน
\newcommand{\SubD}{10 พฤศจิกายน พ.ศ. 2561}

\begin{document}    
    \frontmatter
    \lhead{}\rhead{}\chead{}\lfoot{}\cfoot{\thepage}\rfoot{}
    \makecover    
    \makeinnercover
    \makeengcover
    \makecopyrightcover
    \makethesiscert
    \makeprojectcert
   
    % Setting margin for page numbering on frontmatter
    \newgeometry{top=1in, bottom=1in, left=1.5in, right=1in, includefoot}
    
    \makeabstract{
        ปัญหาความไม่สมดุลของข้อมูล (Class Imbalance Problem) เป็นเรื่องที่ถูกหยิบขึ้นมาวิจัยอย่างแพร่หลาย เนื่องข้อมูลส่วนใหญ่มีจำนวนตัวอย่างของแต่ละกลุ่มข้อมูลไม่เท่ากัน
        ในงานวิจัยชิ้นนี้มุ่งเน้นไปที่การสร้างสรรค์แบบจำลองการเรียนรู้เชิงลึกแบบผสม (Hybrid Deep Learning) ซึ่งเป็นการผสมผสานกันระหว่างแบบจำลองการเรียนรู้เชิงลึก (Deep Learning) หลายตัว 
        มาเพื่อแก้ปัญหาดังกล่าว เนื่องจากแบบจำลอง Hybrid Deep Learning สามารถเพิ่มประสิทธิภาพในการจัดกลุ่มข้อมูลที่มีความสมดุลกันได้ ในทางเดียวกันแบบจำลองดังกล่าวอาจจะสามารถจัดกลุ่มข้อมูลที่ไม่สมดุลกันได้อย่างมีประสิทธิภาพเช่นเดียวกัน 
        ในเบื้องต้นเราได้สร้างแบบจำลอง Deep Learning แบบง่าย แล้วเพ่ิมความซับซ้อนของแบบจำลองขึ้นเรื่อย ๆ เพื่อที่จะได้แบบจำลองที่สมบูรณ์ที่สุด
    }

    \makeabstracteng{
        Classification of imbalanced data is extremely common in practice, and this problem has been widely studied in classical machine learning. 
        A classifier produced from an imbalanced data set is likely to be biased towards the majority class and show inferior classification accuracy on the minority class. 
        This work aims at inventing a hybrid deep learning model for classification on imbalanced data. Recently, many hybrid deep learning model 
        can boost the accuracy on classificaiton of balanced data. In the same way, it may work on imbalanced data effectively. Initially, a simple deep learning model is built and tested. 
        We will find out its disvantages or problems such as overfitting and then solve those by building more complex model. 
        The process will be repeated until we get a perfect model with its satisfying results.
    }

    \makeack

    \newpage
    \addcontentsline{toc}{chapter}{สารบัญ}
    \tableofcontents
    
    \newpage
    \addcontentsline{toc}{chapter}{สารบัญตาราง}
    \listoftables    
    
    \newpage
    \addcontentsline{toc}{chapter}{สารบัญภาพ}
    \listoffigures
    
    % Reset frontmatter page numbering margin, back to original margin from class file
    \restoregeometry

    \mainmatter
    \lhead{}\rhead{\thepage}\chead{}\lfoot{}\cfoot{}\rfoot{}
    
    \chapter{บทนำ}
\label{chapter:introduction}

\section{ที่มาและความสำคัญ}
ในด้านการเรียนรู้ของเครื่องจักร (Machine Learning) ความไม่สมดุลกันของข้อมูล หมายถึง การที่จำนวนตัวอย่างของแต่ละกลุ่มข้อมูลมีจำนวนไม่เท่ากัน 
ซึ่งความไม่สมดุลกันของข้อมูลนี้ถูกนิยามให้เป็นปัญหาในการจัดกลุ่มข้อมูล (Classification) สาเหตุที่ความไม่สมดุลกันของข้อมูลเป็นปัญหา คือ อัลกอริทึมการจัดกลุ่ม จะทำงานได้อย่างมีประสิทธิภาพก็ต่อเมื่อจำนวนตัวอย่างของแต่ละกลุ่มข้อมูลมีจำนวนที่เท่าหรือใกล้เคียงกัน 
เมื่อมีความไม่สมดุลกันของข้อมูลจะทำให้การทำงานของอัลกอริทึมมีประสิทธิภาพด้อยลง ซึ่งอาจจะด้อยลงจนไม่สามารถจัดกลุ่มข้อมูลได้เลย

การจัดกลุ่มข้อมูลที่มีจำนวนตัวอย่างของแต่ละกลุ่มไม่สมดุลกันเป็นเรื่องธรรมดาอย่างมากในทางปฏิบัติ 
เนื่องจากข้อมูลที่เกิดขึ้นล้วนแต่ไม่สามารถคาดเดาได้อย่างแน่นอนว่าจำนวนตัวอย่างของแต่ละกลุ่มจะสมดุลกัน อีกทั้งข้อมูลส่วนใหญ่ยังมีลักษณะที่มีจำนวนตัวอย่างของแต่ละกลุ่มไม่สมดุลกัน 
เช่น ในระหว่างวัวอยู่ในช่วงเป็นสัด ช่วงเวลาที่วัวแสดงพฤติกรรมเป็นสัดจะมีจำนวนน้อยกว่าช่วงเวลาที่วัวไม่แสดงพฤติกรรมเป็นสัด เป็นต้น ในด้านการเรียนรู้ของเครื่องจักร 
มีความเป็นไปได้ว่าตัวจัดกลุ่มข้อมูลที่ถูกสร้างขึ้นจากชุดข้อมูลที่มีจำนวนตัวอย่างของแต่ละกลุ่มไม่สมดุลกันจะมีความลำเอียงในการจัดกลุ่ม กล่าวคือ 
มีโอกาสสูงที่ตัวจัดกลุ่มจะระบุว่าข้อมูลเป็นกลุ่มส่วนมาก (Majority Class) มากกว่าเป็นกลุ่มส่วนน้อย (Minority Class) 
ซึ่งเป็นผลทำให้การระบุข้อมูลเป็นกลุ่มส่วนน้อยมีความแม่นยำที่ต่ำกว่ามาตรฐาน ซึ่งความแม่นยำในการระบุข้อมูลเป็นแต่ละกลุ่มควรจะเท่าหรือใกล้เคียงกัน

ที่ผ่านมาได้มีการศึกษาเกี่ยวกับปัญหาการจัดกลุ่มข้อมูลในลักษณะนี้อย่างกว้างขวาง และได้แสดงให้เห็นถึงความสำคัญของปัญหาของการจัดกลุ่มข้อมูลที่มีจำนวนตัวอย่างของแต่ละกลุ่มไม่สมดุลกัน 
ซึ่งทำให้ประสิทธิภาพการจัดกลุ่มข้อมูลมีความแม่นยำที่ต่ำ ดังนั้นปัญหานี้จำเป็นต้องถูกจัดการ \cite{Buda:2017} เพื่อที่จะแก้ปัญหาการจัดกลุ่มข้อมูลที่มีจำนวนตัวอย่างของแต่ละกลุ่มไม่สมดุลกัน 
ได้มีเทคนิคเกิดขึ้นมากมาย โดยสามารถแบ่งเทคนิคการแก้ปัญหาได้ 2 ระดับ คือ (1) ระดับข้อมูล (Data-Level) 
ที่ซึ่งเป็นการแก้ปัญหาโดยการจัดการข้อมูลก่อนที่จะถูกนำไปประมวลในกระบวนการจัดกลุ่มข้อมูล โดยการสุ่มเพิ่มจำนวนตัวอย่างข้อมูล (Over-Sampling) และการสุ่มลดจำนวนตัวอย่างข้อมูล 
(Under-Sampling) เป็นเทคนิคในการแก้ปัญหาในระดับข้อมูล เทคนิคการแก้ปัญหาในระดับนี้เป็นการแก้ปัญหาแบบเบื้องต้นที่สามารถดำเนินการได้ง่าย 
อย่างไรก็ตามการสุ่มเพิ่มจำนวนตัวอย่างข้อมูลสามารถทำให้เกิดปัญหา Overfitting ตามมาได้อย่างง่ายดาย 
ในทางเดียวกันการสุ่มลดจำนวนตัวอย่างข้อมูลอาจจะเป็นการกำจัดสารสนเทศที่เป็นประโยชน์ต่อการจัดกลุ่มข้อมูลออกไป (2) ระดับตัวจัดกลุ่ม (Classifier-Level) 
ที่ซึ่งเป็นการแก้ปัญหาโดยการจัดการอัลกอริทึมการจัดกลุ่ม โดยการทำเทรสโช (Thresholding) การเรียนรู้แบบความเสียหายที่รู้สึกได้ง่าย (Cost-Sensitive Learning) 
การจัดกลุ่มข้อมูลแบบหนึ่งกลุ่ม (One-Class Classification) และการผนวกกันของหลายเทคนิค อย่างไรก็ตามเทคนิคเหล่านี้มียังไม่สามารถแก้ปัญหาได้อย่างมีประสิทธิภาพในทุก ๆ ชุดข้อมูล 
กล่าวคือ เทคนิคสามารถให้ความแม่นยำในการจัดกลุ่มข้อมูลแต่ละกลุ่มได้อย่างน่าพอใจสำหรับชุดข้อมูล A แต่ไม่สามารถทำได้อย่างมีประสิทธิภาพสำหรับชุดข้อมูล B เป็นต้น 
ดังนั้นเทคนิคใหม่ที่จะสามารถการแก้ปัญหาการจัดกลุ่มข้อมูลที่ไม่สมดุลกันได้อย่างมีประสิทธิภาพ และปรับเข้าได้กับทุกชุดข้อมูลจำเป็นต้องถูกคิดค้นขึ้น

ในปัจจุบันโครงข่ายประสาทเทียมเชิงลึก (Deep Neural Network: DNN) ถูกนำมาใช้ในการจัดกลุ่มข้อมูลอย่างกว้างขวาง จากงานวิจัยที่ผ่านมาได้แสดงให้เห็นว่าแบบจำลอง DNN สามารถจัดกลุ่มได้อย่างมีประสิทธิภาพ อย่างไรก็ตามเมื่อข้อมูลมีความไม่สมดุลกันแบบจำลอง DNN จะไม่สามารถจัดกลุ่มได้อย่างที่ควรจะเป็น~\cite{Buda:2017, Hensman:2015} เนื่องจากการเรียนรู้ที่ไม่มีประสิทธิภาพ สำหรับแบบจำลอง DNN เมื่อเรียนรู้จากข้อมูลที่ไม่สมดุลกัน อาจจะทำให้เกิดความลำเอียงในการเรียนรู้ขึ้น ที่ซึ่งแบบจำลองจะมุ่งเรียนรู้ข้อมูลจากกลุ่มข้อมูลส่วนมากมากกว่าเรียนรู้ข้อมูลจากกลุ่มข้อมูลส่วนน้อย ทำให้แบบจำลองไม่สามารถแยกแยะข้อมูลว่าเป็นของกลุ่มข้อมูลส่วนมากหรือกลุ่มข้อมูลส่วนน้อยได้อย่างมีประสิทธิภาพ 

เพื่อที่จะทำให้แบบจำลอง DNN เรียนรู้ข้อมูลที่ไม่สมดุลกันได้อย่างมีประสิทธิภาพ งานวิจัยจึงได้นำเสนอฟังก์ชันสูญเสีย (Loss Function) แบบใหม่ ที่จะทำให้การเรียนรู้ของแบบจำลองมีความสมดุลกัน โดยฟังก์ชันสูญเสียแบบใหม่นี้ถูกดัดแปลงมาจากฟังก์ชันสูญเสียที่มีอยู่แล้ว ที่ซึ่งจะคำนึงถึงข้อได้เปรียบของแต่ละฟังก์ชันสูญเสีย แล้วนำแนวคิดของฟังก์ชันสูญเสียเหล่านั้นมาผสมผสผานกัน เพื่อสร้างเป็นฟังก์ชันสูญเสียใหม่

\section{ความมุ่งหมายและวัตถุประสงค์ของการศึกษา}
\begin{enumerate}
	\item เพื่อศึกษาการเรียนรู้ของแบบจำลอง DNN ด้วยฟังก์ชันสูญเสียแบบต่าง ๆ
	\item เพื่อคิดค้นฟังก์ชันสูญเสียแบบใหม่ที่ทำให้แบบจำลอง DNN สามารถเรียนรู้จากข้อมูลที่ไม่สมดุลได้อย่างมีประสิทธิภาพ
\end{enumerate}
\section{ขอบเขตการพัฒนาโครงงาน}
\begin{enumerate}
	\item คิดค้นฟังก์ชันสูญเสียแบบใหม่ และทำการทดลองเพื่อเปรียบเทียบประสิทธิภาพของฟังก์ชันสูญเสียแบบใหม่กับฟังก์ชันสูญเสียที่มีอยู่
\end{enumerate}
\section{ขั้นตอนการดำเนินงาน}
\begin{enumerate}
	\item ศึกษาเกี่ยวกับนิยามของความไม่สมดุลกันของข้อมูลในด้านการจัดกลุ่มข้อมูล
	\item ศึกษากระบวนการการเรียนรู้ของแบบจำลอง DNN และการทำงานของฟังก์ชันสูญเสียแบบต่าง ๆ
	\item ศึกษางานวิจัยที่เกี่ยวข้อง
	\item ตั้งข้อสมมติฐาน
	\item ออกแบบการทดลอง
	\item เลือกชุดข้อมูล และ Metrics ที่จะใช้ในการทดลอง
	\item ดำเนินการทำการทดลอง
	\item สรุปผลการทดลอง
\end{enumerate}
\section{ประโยชน์ที่คาดว่าจะได้รับ}
\begin{enumerate}
	\item ฟังก์ชันสูญเสียแบบใหม่ที่มีประสิทธิภาพมากกว่าฟังก์ชันสูญเสียที่มีอยู่
\end{enumerate}


    \chapter{การทบทวนวรรณกรรมที่เกี่ยวข้อง}
\label{chapter:literature-review}

\section{ปัญหาความไม่สมดุลกันของกลุ่มข้อมูล}
ความไม่สมดุลกันของกลุ่มข้อมูล คือ การที่ตัวอย่างของข้อมูลแต่ละกลุ่มมีจำนวนไม่เท่ากัน และจำนวนตัวอย่างนั้นต่างกันมาก เช่น ชุดข้อมูล A มี 2 กลุ่มข้อมูลจากทั้งหมด 10,500 ตัวอย่าง 
แบ่งออกเป็นกลุ่มข้อมูลที่ 1 จำนวน 500 ตัวอย่าง และกลุ่มข้อมูลที่ 2 จำนวน 10,000 เป็นต้น 

ในงานวิจัย~\cite{Anand:1993} ผู้วิจัยได้ทำการศึกษาเกี่ยวกับผลทระทบของความไม่สมดุลกันของข้อมูลในการเรียนรู้ของแบบจำลอง และพบว่าความไม่สมดุลกันของข้อมูลได้ส่งผลกระทบด้านลบต่อกระบวนการ Backpropagation โดยผลกระทบดังกล่าว คือ การที่กลุ่มข้อมูลส่วนมากมีอิทธิพลต่อค่า Gradient ที่จะถูกนำไปใช้ในการปรับค่า Weight มากกว่ากลุ่มข้อมูลส่วนน้อย เนื่องจากจำนวนตัวอย่างของกลุ่มข้อมูลส่วนมากในแต่ละ Batch ของการเรียนรู้ นั้นมีมากกว่า ทำให้ค่าสูญเสียรวมมีลักษณะที่ค่าสูญของกลุ่มข้อมูลส่วนมากไปกลบค่าสูญเสียของกลุ่มข้อมูลส่วนน้อย 

เหตุการณ์ดังกล่าวทำให้ลักษณะของการเรียนรู้ของแบบจำลองมุ่งไปที่การเรียนรู้เฉพาะกลุ่มข้อมูลส่วนมาก  กล่าวคือ ค่าสูญเสียของกลุ่มข้อมูลส่วนมากจะลดลงอย่างรวดเร็ว ในขณะที่ค่าสูญเสียของกลุ่มข้อมูลส่วนน้อยเพิ่มขึ้นเรื่อย ๆ ในช่วงต้นของการเรียนรู้ สุดท้ายทำให้การเรียนรู้ของแบบจำลองเข้าลู่เข้าจุดที่ดีที่สุดช้าหรือไม่สามารถเรียนรู้ที่จะจัดกลุ่มได้เลย

ความไม่สมดุลกันของกลุ่มข้อมูลนั้นมีอยู่ 2 ประเภท คือ Step Imbalance~\cite{Buda:2017} และ Long-Tailed Imbalance~\cite{Liu:2019} ตามรายละเอียดดังนี้

\begin{itemize}
  \item Step Imbalance เป็นความไม่สมดุลกันของกลุ่มข้อมูลที่จำนวนตัวอย่างของกลุ่มข้อมูลส่วนน้อยแต่ละกลุ่มมีจำนวนเท่ากัน และจำนวนตัวอย่างของกลุ่มข้อมูลส่วนมากแต่ละกลุ่มมีจำนวนเท่ากัน โดยอัตราส่วนของกลุ่มของส่วนน้อยและส่วนมาก ($\mu$) สามารถคำนวณได้จากสมการที่ \ref{eq:stepimbalance} ตัวอย่างการกระจายของจำนวนตัวอย่างของกลุ่มข้อมูลแสดงดังรูปที่ \ref{fig:imb-dist:1} และ \ref{fig:imb-dist:2} 
  \begin{equation}
      \mu = \frac{\left | \left \{ i\in  \left \{ 1,...,N \right \}: C_{i}\:is\:minority\:class \right \} \right |}{N},
      \label{eq:stepimbalance}
    \end{equation}

    โดยที่ $C_{i}$ คือ ชุดของตัวอย่างของกลุ่มข้อมูล $i$ และ N คือ จำนวนกลุ่มตัวอย่างทั้งหมด
  \item Long-Tailed Imbalance เป็นความไม่สมดุลกันของกลุ่มข้อมูลที่จำนวนตัวอย่างของแต่ละกลุ่มข้อมูลมีจำนวนไม่เท่ากันตามตัวอย่างการกระจายของจำนวนตัวอย่างของกลุ่มข้อมูลในรูปที่ \ref{fig:imb-dist:3}
  
\end{itemize}

สามารถคำนวณค่าอัตราส่วนระหว่างจำนวนตัวอย่างของกลุ่มข้อมูลส่วนน้อยและกลุ่มข้อมูลส่วนมาก ($p$) ได้ตามสมการที่ \ref{eq:p_ratio}

\begin{equation}
  p = \frac{max_{i}\left \{ \left | C_{i} \right | \right \}}{min_{i}\left \{ \left | C_{i} \right | \right \}}
  \label{eq:p_ratio}
\end{equation}

\begin{figure}[h]
  \centering
  \subfigure[]{
      \includegraphics[width=0.45\columnwidth]{imbalance-distribution-1}
      \label{fig:imb-dist:1}
  }
  \subfigure[]{
      \includegraphics[width=0.45\columnwidth]{imbalance-distribution-2}
      \label{fig:imb-dist:2}
  }
  \subfigure[]{
    \includegraphics[width=0.45\columnwidth]{imbalance-distribution-3}
    \label{fig:imb-dist:3}
  }
  \caption{ตัวอย่างการกระจายของจำนวนตัวอย่างของกลุ่มข้อมูล (ก) $p = 10$, $\mu = 0.5$ (ข) $p = 2$, $\mu = 0.9$ (ค) $p = 10$}
  \label{fig:imb-dist}
\end{figure}
\FloatBarrier

เพื่อที่จะพิสูจน์ว่าวิธีการที่นำเสนอในงานวิจัยนี้นั้นมีประสิทธิภาพ ในเบื้องต้นจะมุ่งศึกษาที่การจัดกลุ่มข้อมูลสองกลุ่ม ที่ซึ่งลักษณะของความไม่สมดุลกันจะเป็นแบบ Step Imbalance และในอนาคตจะทำการพิสูจน์วิธีการที่นำเสนอกับการจัดกลุ่มข้อมูลหลายกลุ่ม โดยในรายงานฉบับนี้จะขอกล่าวถึงกลุ่มข้อมูลส่วนมากว่า Negative Class และกลุ่มข้อมูลส่วนน้อยว่า Positive Class

\section{ฟังก์ชันสูญเสีย (Loss Function)}
ในกระบวนการเรียนรู้ของแบบจำลอง DNN ฟังก์ชันสูญเสียจะถูกใช้ในการคำนวณค่าสูญเสีย เพื่อนำไปปรับค่า Weight ของแบบจำลอง เมื่อค่าสูญเสียยิ่งมาก ค่า Weight จะถูกปรับจากค่าเดิมมาก ในทางเดียวกันถ้าค่าสูญเสียน้อย ค่า Weight จะถูกปรับจากค่าเดิมน้อยเช่นกัน หรือก็คือแบบจำลองเริ่มไม่เรียนรู้อะไรเพิ่มเติมแล้ว ดังนั้นค่าสูญเสียต้องเป็นค่าที่เหมาะสมให้มากที่สุด ไม่เช่นนั้นจะทำให้การปรับค่า Weight ของแบบจำลองเกิดการคลาดเคลื่อนได้ อย่างไรก็ตามปัญหานี้จะพบได้ในเฉพาะการเรียนรู้ของแบบจำลองกับข้อมูลที่ไม่สมดุล ดังตัวอย่างการคำนวณค่าสูญเสียใน~\cite{Wang:2016} โดยที่จำนวนตัวอย่างของกลุ่มข้อมูลส่วนมากและส่วนน้อยเท่ากับ 90 และ  10 ตัวอย่างตามลำดับ และแบบจำลองทำนายข้อมูลของกลุ่มข้อมูลส่วนมากผิดไป 4 ตัวอย่าง และสำหรับกลุ่มข้อมูลส่วนน้อยผิดไป 5 ตัวอย่าง ถ้าคำนวณค่าสูญเสียด้วย Mean Squared Error (MSE) จะได้ค่าสูญเสียเท่ากับ 0.09 ซึ่งจากค่าสูญเสียดังกล่าว มันไม่สมเหตุสมผลเลยที่ค่าสูญเสียจะน้อยขนาดนี้ เพราะแบบจำลองทำนายข้อมูลของกลุ่มข้อมูลส่วนน้อยผิดไปตั้งครึ่งหนึ่ง ด้วยเหตุนี้ทำให้ในการเรียนรู้ของแบบจำลองในรอบถัดไป ถูกกำหนดให้เปลี่ยนแปลงค่า Weight ไม่มาก ดังนั้นจากปัญหาที่กล่าวมาจำเป็นจะต้องมีฟังก์ชันสูญเสียที่สามารถคำนวณค่าสูญเสียได้อย่างสมเหตุสมผลที่สุด เพื่อให้การเรียนรู้ของแบบจำลองนั้นมีประสิทธิภาพ
    \chapter{วิธีการดำเนินการวิจัย}
\label{chapter:experiment}

\section{Hybrid Loss}
\subsection{Focal Loss}
สิ่งดลใจในการคิดค้น Focal Loss (FL)~\cite{Lin:2017} คือ Cross Entropy ไม่สามารถควบคุมความสมดุลระหว่างค่าสูญเสียจากกลุ่มข้อมูลส่วนมากและค่าสูญเสียจากกลุ่มข้อมูลส่วนน้อยได้ เพราะข้อมูลจากทั้งสองกลุ่มไม่สมดุลกัน แม้ว่าการใส่ Weighting Factor ($\alpha$) จะสามารถแก้ปัญหานี้ได้ในเบื้องต้น แต่มันก็ไม่สามารถที่จะแยกความแตกต่างระหว่าง ตัวอย่างที่ง่าย (ตัวอย่างที่ให้ค่าสูญเสียน้อย) และ ตัวอย่างที่ยาก (ตัวอย่างที่ให้ค่าสูญเสียมาก) ได้ ซึ่งตัวอย่างที่ง่ายของกลุ่มข้อมูลส่วนมากจะส่วนในการคำนวณค่าสูญเสียรวมมาก ทำให้มีอิทธิพลต่อการคำนวณค่า Gradient มากเช่นเดียวกัน ซึ่งโดยปกติแล้วตัวอย่างที่ยากของกลุ่มข้อมูลส่วนมากจะประกอบไปด้วยข้อมูลที่เป็นประโยชน์ต่อการจัดกลุ่มมากกว่าตัวอย่างที่ง่ายของ
กลุ่มข้อมูลส่วนน้อย~\cite{Zhu:2019} ดังนั้นการเรียนรู้จากตัวอย่างที่ยากของกลุ่มข้อมูลส่วนมากจะมีประสิทธิภาพมากกว่าการเรียนรู้จากตัวอย่างที่ง่ายของกลุ่มข้อมูลส่วนมาก

จากที่กล่าวมาข้างต้นทำให้มีความจำเป็นที่จะต้องลดการมีส่วนร่วมในการคำนวณค่าสูยเสียรวมของตัวอย่างที่ง่าย และให้ความสนใจที่การมีส่วนร่วมในการคำนวณค่าสูยเสียรวมของตัวอย่างที่ยาก ดังนั้น FL ถูกออกแบบมาเพื่อการนี้โดยการเพิ่ม Modulating Factor ($(1 - p_{t})^{\gamma}$) เข้าไปใน Cross Entropy เพื่อที่จะลดน้ำหนักของการเรียนรู้จากตัวอย่างที่ง่าย ที่ซึ่ง Modulating Factor จะช่วยลดการมีส่วนร่วมในการคำนวณค่าสูญเสียรวมจากตัวอย่างที่ง่ายและมุ่งไปที่การเรียนรู้จากตัวอย่างที่ยาก สำหรับการคำนวณค่าสูญเสียและค่าสูญเสียรวมของ FL สามารถคำนวณได้ตามสมการที่~\ref{eq:fl} และ~\ref{eq:tfl} ตามลำดับ

\begin{equation} \label{eq:fl}
FL(p_{t}) = - \alpha_{i}(1 - p_{t})^\gamma \log (p_{t})
\end{equation}

\begin{equation} \label{eq:tfl}
    l_{FL} = \frac{1}{n} \sum_{i = 1}^{n} -\alpha_{i}(1 - p_{t}^{i})^{\gamma}\log (p_{t}^{i})
\end{equation}

สมการด้านบนเป็น FL ในรูปแบบที่เพิ่ม Weighting Factor เข้ามาด้วยเพื่อควบคุมความสมดุลระหว่างค่าสูญเสียจากกลุ่มข้อมูลส่วนมากและค่าสูญเสียจากกลุ่มข้อมูลส่วนน้อย โดยที่ $\gamma$ คือ Focusing Parameter และสำหรับค่าของ $p_{t}$ สามารถถูกระบุได้ดังสมการที่~\ref{eq:p}

\begin{equation} \label{eq:p}
	p_{t} = 
	\begin{cases}
		p & \text{ if } y = 1\\ 
		1 - p & \text{ otherwise, } 
	\end{cases}
\end{equation} 

โดยที่ $p$ คือ ค่าความน่าจะเป็นในการทำนายของโมเดล ดังนั้น $p_{t}^{i}$ คือ ค่า $p_{t}$ ของตัวอย่าง $i$

ในทางปฎิบัติ $\alpha_{i}$ จะมีค่าเท่ากับ $\alpha$ ถ้ากลุ่มข้อมูลของตัวอย่าง $i$ คือ กลุ่มข้อมูลส่วนน้อย และจะมีค่าเท่ากับ $1-\alpha$ ถ้ากลุ่มข้อมูลของตัวอย่าง $i$ คือ กลุ่มข้อมูลส่วนมาก

\subsection{Mean False Error}
Mean False Error (MFE)~\cite{Wang:2016} เป็นฟังก์ชันสูญเสียที่ถูกแก้ไขมาจาก Mean Squared Error (MSE) สิ่งจูงใจในการคิดค้นฟังก์ชันสูญเสียนี้ขึ้นมา คือ MSE ไม่สามารถตรวจจับค่าสูญเสียจากกลุ่มข้อมูลส่วนน้อยได้อย่างมีประสิทธิภาพ พูดง่าย ๆ คือ ค่าสูญเสียรวมที่คำนวณด้วย MSE มันจะมาจากค่าเฉลี่ยของค่าสูญเสียของข้อมูลทั้งหมด โดยสมการคำนวณค่าสูญเสียของ MSE และ สมการคำนวณค่าสูญเสียรวม แสดงดังสมการที่~\ref{eq:mse} และ~\ref{eq:tmse} ตามลำดับ

\begin{equation} \label{eq:mse}
    MSE = \frac{1}{2}(y - d)^{2}
\end{equation}

\begin{equation} \label{eq:tmse}
    l_{MSE} = \frac{1}{n}\sum_{i=1}^{n}\frac{1}{2}(y^{i} - d^{i})^{2}
\end{equation}

จากสมการด้านบน $l$ คือ ค่าสูญเสียรวม, $n$ คือ จำนวนตัวอย่างทั้งหมด, $y^{i}$ ค่ากลุ่มข้อมูลจริงของตัวอย่าง $i$ และ $d^{i}$ คือ ค่าทำนายของโมเดลของตัวอย่าง $i$ โดยที่ $d$ สามารถคำนวณได้จากฟังก์ชัน Logistic ใด ๆ เช่น ฟังก์ชัน Sigmoid ดังสมการที่~\ref{eq:sigmoid} เป็นต้น

\begin{equation} \label{eq:sigmoid}
d =  \frac{\mathrm{1} }{\mathrm{1} + e^{-x} }
\end{equation}

โดยที่ $x$ คือ เอ๊าต์พุตจาก Layer ก่อนหน้า

จากคำนวณค่าสูญเสียรวมด้วย MSE นั้นหมายความว่าค่าสูญเสียจากกลุ่มข้อมูลส่วนมากจะไปกลบค่าสูญเสียจากกลุ่มข้อมูลส่วนน้อย เนื่องจากจำนวนตัวอย่างของกลุ่มข้อมูลส่วนมากนั้นมีมากกว่า เพื่อที่จะแก้ปัญหานี้ MFE ถูกออกแบบให้สามารถคำนวณค่าสูญเสียรวมด้วยการรวมกันระหว่าง ค่าสูญเสียเฉลี่ยของกลุ่มข้อมูลส่วนน้อย กับ ค่าสูญเสียเฉลี่ยของกลุ่มข้อมูลส่วนมาก ตามสมการที่~\ref{eq:mfe} ซึ่ง

\begin{equation} \label{eq:mfe}
    l_{MFE} = \frac{1}{n\_major}\sum_{i=1}^{n\_major}\frac{1}{2}(y^{i} - d^{i})^{2} +\frac{1}{n\_minor}\sum_{i=1}^{n\_minor}\frac{1}{2}(y^{i} - d^{i})^{2}
\end{equation}

โดยที่ $n\_major$ คือ จำนวนตัวอย่างของกลุ่มข้อมูลส่วนมาก และ $n\_minor$ คือ จำนวนตัวอย่างของกลุ่มข้อมูลส่วนน้อย

ผลการทดลองใช้ MFE ในการเรียนรู้ของโมเดลกับชุดข้อมูล CIFAR-100 แสดงให้เห็นว่า MFE สามารถให้ผลการทำนายที่แม่นยำมากกว่า MSE อย่างสิ้นเชิง~\cite{Wang:2016}

\subsection{นิยามของ Hybrid Loss}
คุณสมบัติที่สำคัญของ FL คือ มันสามารถควบความแตกต่างระหว่างตัวอย่างที่ง่ายและตัวอย่างที่ยาก ยิ่งไปกว่านั้นการที่เพิ่ม $\alpha$ เข้าไปในการคำนวณค่าสูญเสียยังช่วยทำให้ความสำคัญของตัวอย่างของกลุ่มข้อมูลส่วนน้อยและส่วนมากมีความสมดุลกัน อย่างไรก็ตามในเมื่อค่าสูญเสียรวมคือค่าเฉลี่ยของค่าสูญเสียของข้อมูลทั้งหมด มันก็ยังมีโอกาสที่ค่าสูญเสียจากกลุ่มข้อมูลส่วนมากจะไปกลบค่าสูญเสียจากกลุ่มข้อมูลส่วนน้อย ถ้าเราแก้ปัญหาด้วยการกำหนดค่า $\alpha$ ให้มีค่าที่มาก ก็อาจจะช่วยแก้ปัญหานี้ได้ในเบื้องต้น ผลการทดลองใน~\cite{Lin:2017} ได้แสดงให้เห็นว่าการที่ $\alpha$ มีค่าที่มากก็ไม่ทำให้ประสทธิภาพของโมเดลดีไปกว่าค่าที่น้อยกว่าเลย ดังนั้นการเพิ่ม $\alpha$ เข้าไปในการคำนวณค่าสูญเสียอาจจะไม่เพียงพอในการแก้ปัญหานี้

เพื่อที่จะควบคุมค่าสูญเสียจากกลุ่มข้อมูลส่วนมากและกลุ่มข้อมูลส่วนน้อยให้มีความสมดุลกันอย่างสิ้นเชิง
ในงานวิจัยนี้จึงได้เสนอที่จะนำรูปแบบการคำนวณค่าสูญเสียรวมของ MFE มาประยุกต์ใช้ร่วมกับการคำนวณค่าสูญเสียของ FL กล่าวคือ การคำนวณค่าสูญเสียของแต่ละตัวอย่างยังคงเหมือนเดิมตามการคำนวณด้วย FL แต่ในการคำนวณค่าสูญเสียรวมจะเป็นการนำค่าสูญเสียเฉลี่ยของกลุ่มข้อมูลส่วนมากและค่าสูญเสียเฉลี่ยของกลุ่มข้อมูลส่วนน้อยมาบวกกัน ดังสมการที่~\ref{eq:hybird1}

\begin{equation} \label{eq:hybird1}
    l_{Hybrid} = \frac{1}{n\_major} \sum_{i = 1}^{n\_major} -\alpha_{i}(1 - p_{t}^{i})^{\gamma}\log (p_{t}^{i}) + \frac{1}{n\_minor} \sum_{i = 1}^{n\_minor} -\alpha_{i}(1 - p_{t}^{i})^{\gamma}\log (p_{t}^{i})
\end{equation}

\subsection{การหาอนุพันธ์}

\subsubsection{อนุพันธ์ของ FL}
กำหนดให้มี $x$ และ $y$ โดยที่ $y$ คือ ค่ากลุ่มข้อมูลจริง สมมติให้

\begin{equation}
	p = \sigma (x) = \frac{1}{1 + e^{-x}}
\end{equation}

ดังนั้น จากสมการที่~\ref{eq:p} จะได้

\begin{equation}
	p_{t} = \frac{1}{1 + e^{xy}}
\end{equation}

เมื่อหาอนุพันธ์ของ $p_{t}$ เทียบกับ $x$ จะได้

\begin{equation}
\frac{\partial p_{t}}{\partial x} = y(1 - p_{t})p_{t}
\end{equation}

กำหนดให้

\begin{equation}
    l_{FL} = \frac{1}{n} \sum_{i = 1}^{n} - (1 - p_{t}^{i})^{\gamma}\log (p_{t}^{i})
\end{equation}

โดยที่ $\gamma$ เป็นค่าคงที่ เมื่อหาอนุพันธ์ของ $l_{FL}$ เทียบกับ $x$ จะได้

\begin{equation}
\begin{split}
\frac{\partial l_{FL}}{\partial x^{i}} & = \frac{\partial l_{FL}}{\partial p_{t}^{i}} * \frac{\partial p_{t}^{i}}{\partial x^{i}} \\
& =\frac{1}{n} \sum_{i = 1}^{n} [\gamma (1 - p_{t}^{i})^{\gamma - 1} \log (p_{t}^{i}) + \frac{(1 - p_{t}^{i})^{\gamma}}{p_{t}^{i}}] * [y^{i}(1 - p_{t}^{i})p_{t}^{i}]  \\
& = \frac{1}{n} \sum_{i = 1}^{n} y^{i} (1 - p_{t}^{i})^{\gamma} (\gamma p_{t}^{i} \log (p_{t}^{i}) + p_{t}^{i} - 1)
\end{split}
\end{equation}

\subsubsection{อนุพันธ์ของ MFE}
จากสมการที่~\ref{eq:sigmoid} สามารถหาอนุพันธ์ของ $d$ เทียบกับ $x$ ได้ดังนี้

\begin{equation}
\frac{\partial d}{\partial x} = d(1 - d)
\end{equation}

กำหนดให้

\begin{equation}
MSE_{major} = \frac{1}{n\_major}\sum_{i=1}^{n\_major}\frac{1}{2}(y^{i} - d^{i})^{2}
\end{equation}

\begin{equation}
MSE_{minor} = \frac{1}{n\_minor}\sum_{i=1}^{n\_minor}\frac{1}{2}(y^{i} - d^{i})^{2}
\end{equation}

ดังนั้น

\begin{equation}
    l_{MFE} = MSE_{major} + MSE_{minor}
\end{equation}

เมื่อหาอนุพันธ์ของ $l_{MFE}$ เทียบกับ $x$ จะได้

\begin{equation}
\frac{\partial l_{MFE}}{\partial x^{i}} = \frac{\partial MSE_{major}}{\partial x^{i}} + \frac{\partial MSE_{minor}}{\partial x^{i}}
\end{equation}

โดยที่

\begin{equation} \label{eq:fpe_derivative}
\begin{split}
\frac{\partial MSE_{major}}{\partial x^{i}} & = \frac{\partial MSE_{major}}{\partial d^{i}} * \frac{\partial d^{i}}{\partial x^{i}} \\
& = - \frac{1}{n\_major} \sum_{i=1}^{n\_major} (y^{i} - d^{i}) d^{i} (1 - d^{i})
\end{split}
\end{equation}

\begin{equation} \label{eq:fne_derivative}
\begin{split}
\frac{\partial MSE_{minor}}{\partial x^{i}} & = \frac{\partial MSE_{minor}}{\partial d^{i}} * \frac{\partial d^{i}}{\partial x^{i}} \\
& = - \frac{1}{n\_minor} \sum_{i=1}^{n\_minor} (y^{i} - d^{i}) d^{i} (1 - d^{i})
\end{split}
\end{equation}

เราจะใช้อนุพันธ์ที่ต่างกันสำหรับตัวอย่างของแต่ละกลุ่มข้อมูล กล่าวคือ ถ้าเป็นตัวอย่างของกลุ่มข้อมูลส่วนมาก สมการที่~\ref{eq:fpe_derivative} จะถูกใช้ และสมการที่~\ref{eq:fne_derivative} จะถูกใช้เมื่อตัวย่างเป็นของกลุ่มข้อมูลส่วนน้อย



\subsubsection{อนุพันธ์ของ Hybrid Loss}
กำหนดให้

\begin{equation}
    FL_{major} = \frac{1}{n\_major} \sum_{i = 1}^{n\_major} - (1 - p_{t}^{i})^{\gamma}\log (p_{t}^{i})
\end{equation}

\begin{equation}
    FL_{minor} = \frac{1}{n\_minor} \sum_{i = 1}^{n\_minor} - (1 - p_{t}^{i})^{\gamma}\log (p_{t}^{i})
\end{equation}

ดังนั้น

\begin{equation}
    l_{Hybrid} = FL_{major} + FL_{minor}
\end{equation}

เมื่อหาอนุพันธ์ของ $l_{Hybrid}$ เทียบกับ x จะได้

\begin{equation}
\frac{\partial l_{Hybrid}}{\partial x^{i}} = \frac{\partial FL_{major}}{\partial x^{i}} + \frac{\partial FL_{minor}}{\partial x^{i}}
\end{equation}

โดยที่

\begin{equation} \label{fl_major_derivative}
\begin{split}
\frac{\partial FL_{major}}{\partial x^{i}} & = \frac{\partial FL_{major}}{\partial p_{t}^{i}} * \frac{\partial p_{t}^{i}}{\partial x^{i}} \\
& = \frac{1}{n\_major} \sum_{i = 1}^{n\_major} y^{i} (1 - p_{t}^{i})^{\gamma} (\gamma p_{t}^{i} \log (p_{t}^{i}) + p_{t}^{i} - 1)
\end{split}
\end{equation}

\begin{equation} \label{fl_minor_derivative}
\begin{split}
\frac{\partial FL_{minor}}{\partial x^{i}} & = \frac{\partial FL_{minor}}{\partial p_{t}^{i}} * \frac{\partial p_{t}^{i}}{\partial x^{i}} \\
& = \frac{1}{n\_minor} \sum_{i = 1}^{n\_minor} y^{i} (1 - p_{t}^{i})^{\gamma} (\gamma p_{t}^{i} \log (p_{t}^{i}) + p_{t}^{i} - 1)
\end{split}
\end{equation}

เช่นเดียวกับการใช้อนุพันธ์ของ MFE ก็คือ เราจะใช้สมการที่~\ref{fl_major_derivative} ถ้าตัวอย่างเป็นของกลุ่มข้อมูลส่วนมาก และจะใช้สมการที่~\ref{fl_minor_derivative} ถ้าตัวอย่างเป็นของกลุ่มข้อมูลส่วนน้อย

\section{Metrics สำหรับการประเมินประสิทธิภาพของโมเดล}
\subsection{F1-Score}
F1-Score เป็นค่าเฉลี่ยของ Precision และ Recall ที่ซึ่งเราสามารถพิจารณาประสิทธิภาพของแบบจำลองด้วย F1-Score แทนการพิจารณาด้วย Precision หรือ Recall โดย F1-Score สามารถคำนวณได้จากสมการที่ \ref{eq:f1-score}

\begin{equation}
  F1 = 2 * \left ( \frac{precision * recall}{precision + recall} \right )
  \label{eq:f1-score}
\end{equation}

โดยที่ Precision และ Recall สามารถคำนวณได้จากสมการที่ \ref{eq:precision} และ \ref{eq:recall} ตามลำดับ

\begin{equation}
  precision = \frac{TP}{TP + FP}
  \label{eq:precision}
\end{equation}

\begin{equation}
  recall = \frac{TP}{TP + FN}
  \label{eq:recall}
\end{equation}

โดยที่

\begin{itemize}
  \item True Positive (TP) คือ จำนวนตัวอย่างที่แบบจำลองตรวจจับได้ว่าเป็น Positive Class และตัวอย่างเหล่านั้นเป็น Positive Class
  \item True Negative (TN) คือ จำนวนตัวอย่างที่แบบจำลองตรวจจับได้ว่าเป็น Negative Class และตัวอย่างเหล่านั้นเป็น Negative Class
  \item False Positive (FP) คือ จำนวนตัวอย่างที่แบบจำลองตรวจจับได้ว่าเป็น Positive Class แต่ตัวอย่างเหล่านั้นเป็น Negative Class
  \item False Negative (FN) คือ จำนวนตัวอย่างที่แบบจำลองตรวจจับได้ว่าเป็น Negative Class แต่ตัวอย่างเหล่านั้นเป็น Positive Class
\end{itemize}

\subsection{Receiver Operating Characteristic Curve (ROC Curve)}
ROC Curve เป็นกราฟความสัมพันธ์ระหว่าง True Positive Rate (TPR) และ False Positive Rate (FPR) โดย TPR หรือ Recall คือ 
ความน่าจะเป็นที่แบบจำลองสามารถตรวจจับ Positive Class จากจำนวน Positive Class ทั้งหมด และ FPR คือ 
ความน่าจะเป็นที่แบบจำลองจะตรวจจับ Positive Class จากจำนวน Negative Class ทั้งหมด ทั้ง TPR และ FPR สามารถคำนวณได้จากสมการที่ \ref{eq:tpr} และ \ref{eq:fpr} ตามลำดับ

\begin{equation}
  TPR = recall = \frac{TP}{TP + FN}
  \label{eq:tpr}
\end{equation}

\begin{equation}
  FPR = \frac{FP}{FP + TN}
  \label{eq:fpr}
\end{equation}

\subsection{Area Under the ROC Curve (AUC)}
AUC คือความน่าจะเป็นที่แบบจำลองจะระบุตัวอย่างของ Positive Class ว่าเป็น Positive Class และตัวอย่างของ Negative Class ว่าเป็น Negative Class 
โดยถ้าค่า AUC เข้าใกล้ 1 นั้นหมายความว่าแบบจำลองมีความสามารถในการแยก Positive Class ออกจาก Negative Class ได้เป็นอย่างดี 
ในทางเทคนิค AUC ก็คือพื้นที่ใต้กราฟของ ROC Curve โดยความสัมพันธ์กันระหว่างค่า AUC และ ROC Curve ถูกแสดงดังรูปที่ \ref{fig:roc}

\begin{figure}[h]
  \centering
  \subfigure[]{
      \includegraphics[width=0.45\columnwidth]{roc-1.png}
      \label{fig:roc:1}
  }
  \subfigure[]{
      \includegraphics[width=0.45\columnwidth]{roc-2.png}
      \label{fig:roc:3}
  }
  \subfigure[]{
    \includegraphics[width=0.45\columnwidth]{roc-3.png}
    \label{fig:roc:3}
  }
  \subfigure[]{
    \includegraphics[width=0.45\columnwidth]{roc-4.png}
    \label{fig:roc:3}
  }
  \caption{ตัวอย่างกราฟ ROC Curve แบบต่าง ๆ}
  \label{fig:roc}
\end{figure}
\FloatBarrier

   	\chapter{ผลการทดลอง}
\label{chapter:result}

ในงานวิจัยนี้การทดลองถูกแบ่งออกเป็น 2 ส่วนหลัก ๆ คือ การจัดกลุ่มข้อมูลสองกลุ่ม (Binary Classification) และ การจัดกลุ่มข้อมูลหลายกลุ่ม (Multi-Class Classification) โดยประเภทของข้อมูลที่ถูกใช้ในการทดลองมี 2 ประเภท คือ ข้อมูลแบบมีโครงสร้าง และ ข้อมูลภาพ สำหรับข้อมูลแบบมีโครงสร้างจะมีทั้งหมด 23 ชุดข้อมูล ที่ซึ่งแต่ละชุดข้อมูลจะมีกลุ่มของข้อมูลอยู่ 2 กลุ่ม โดยชุดข้อมูลทั้งหมดนี้จะเป็นชุดข้อมูลที่ไม่สมดุลสำหรับการวัดเปรียบเทียบสมรรถนะเกณฑ์มาตรฐาน (Benchmark) ตามที่ถูกนำเสนอไปใน~\cite{Ding:2011} สำหรับข้อมูลภาพจะมี 2 ชุดข้อมูล คือ CIFAR-100~\cite{Krizhevsky:2009} และ CIFAR-10~\cite{Krizhevsky:2009} โดยทั้งสองนี้เป็นชุดข้อมูลที่สมดุลกันอยู่แล้ว และมีจำนวนกลุ่มข้อมูล 100 และ 10 กลุ่มตามลำดับ แต่จะถูกนำมาดัดแปลงให้ไม่สมดุลเพื่อทำการทดลอง โดยรายละเอียดของการดัดแปลงของ CIFAR-100 และ CIFAR-10 จะถูกอธิบายไว้ในหัวข้อย่อยต่อไป

สำหรับสถาปัตยกรรมของโมเดลที่ใช้ในการทดลองถูกแบ่งออกเป็น 2 ส่วน คือ โมเดลโครงข่ายประสาทเทียมแบบมี 1 Hidden Layer สำหรับการทดลองกับข้อมูลแบบมีโครงสร้าง และ โมเดล ResNet~\citep{He:2016} สำหรับการทดลองกับข้อมูลภาพ

\section{Binary Classification}
\subsubsection{ผลการทดลองกับชุดข้อมูลแบบมีโครงสร้าง}
\subsubsection{ผลการทดลองกับชุดข้อมูล CIFAR-100}
การทดลองใน~\cite{Wang:2016} ได้ทำการดัดแปลง CIFAR-100 โดยการแบ่งชุดข้อมูลออกเป็น 3 ชุดข้อมูลย่อย คือ Tree1, Tree2 และ Household และเลือกข้อมูลมา 2 กลุ่มที่แตกต่างกันสำหรับแต่ละชุดข้อมูลย่อย ที่ซึ่งข้อมูลในแต่ละชุดข้อมูลย่อยจะถูกทำให้ไม่สมดุลกัน สำหรับรายละเอียดของกลุ่มข้อมูลในแต่ละชุดข้อมูลย่อยดังนี้

\paragraph{Tree1}
เป็นชุดข้อมูลที่ประกอบด้วยข้อมูลของกลุ่ม \emph{maple tree} และ \emph{oak tree} โดยจะให้กลุ่ม \emph{maple tree} เป็นกลุ่มข้อมูลส่วนมาก และ \emph{oak tree} เป็นกลุ่มข้อมูลส่วนน้อย

\paragraph{Tree2}
เป็นชุดข้อมูลที่ประกอบด้วยข้อมูลของกลุ่ม \emph{maple tree} และ \emph{palm tree} โดยจะให้กลุ่ม \emph{maple tree} เป็นกลุ่มข้อมูลส่วนมาก และ \emph{palm tree} เป็นกลุ่มข้อมูลส่วนน้อย

\paragraph{Household}
เป็นชุดข้อมูลที่ประกอบด้วยข้อมูลของกลุ่ม \emph{household furniture} และ \emph{household electrical devices} โดยจะให้กลุ่ม \emph{household furniture} เป็นกลุ่มข้อมูลส่วนมาก และ \emph{household electrical devices} เป็นกลุ่มข้อมูลส่วนน้อย

รูปที่~\ref{fig:result-cifar100} แสดงผลการทดลองของแต่ละชุดข้อมูล โดยจากผลการทดลองสามารถสรุปได้ว่าฟังก์ชันสูญเสียแบบ Hybrid มีประสิทธิภาพเหนือกว่าฟังก์ชันสูญเสียอื่น ๆ อย่างสิ้นเชิง

\begin{figure}[h]
  \centering
  \subfigure[]{
      \includegraphics[width=0.45\columnwidth]{results/Household_cifar100.png}
      \label{fig:result-cifar100-household}
  }
  \subfigure[]{
      \includegraphics[width=0.45\columnwidth]{results/Tree1_cifar100.png}
      \label{fig:result-cifar100-tree1}
  }
  \subfigure[]{
    \includegraphics[width=0.45\columnwidth]{results/Tree2_cifar100.png}
    \label{fig:result-cifar100-tree2}
  }
  \caption{ประสิทธิภาพของแต่ละฟังก์ชันสูญเสีย เมื่อทดสอบกับชุดข้อมูลที่มีอัตราจำนวนตัวอย่างของกลุ่มข้อมูลส่วนน้อยที่แตกต่างกัน}
  \label{fig:result-cifar100}
\end{figure}
\FloatBarrier

\label{ex:cifar-100}
\section{Multi-Class Classification}
\subsubsection{ผลการทดลองกับชุดข้อมูล Long-Tailed-Imbalanced CIFAR-10}
ชุดข้อมูล CIFAR-10 จะถูกดัดแปลงให้ไม่สมดุลแบบ Long-Tail
\label{ex:long-tailed-cifar-10}
\subsubsection{ผลการทดลองกับชุดข้อมูล Step-Imbalanced CIFAR-10}
ชุดข้อมูล CIFAR-10 จะถูกดัดแปลงให้ไม่สมดุลแบบ Step
\label{ex:step-cifar-100}
   	\chapter{บทสรุป}
\label{chapter:conclusion}

ในงายวิจัยนี้ ผู้วิจัยได้ศึกษาการทำงานของฟังก์ชันสูญเสียแบบ Mean False Error (MFE) และ Focal Loss (FL) สำหรับการเรียนรู้ของแบบจำลองโครงข่ายประสาทเทียมเชิงลึกอย่างละเอียด เพื่อที่จะหาข้อได้เปรียบของแต่ละฟังก์ชัน ฟังก์ชันสูญเสียทั้งสองดังกล่าวเป็นฟังก์ชันสูญเสียที่ถูกออกแบบมาเพื่อจัดการกับปัญหาความไม่สมดุลกันของข้อมูล
จากผลการศึกษาพบว่าแต่ละฟังก์ชันสูญเสียมีวิธีการจัดการปัญหาความไม่สมดุลกันของข้อมูลที่ต่างกัน และแนวคิดของแต่ละฟังก์ชันสามารถนำมารวมกันได้ เพื่อที่จะใช้ข้อได้เปรียบของทั้งสองฟังก์ชันเพิ่มประสิทธิภาพการเรียนรู้ของโมเดล ด้วยเหตุนี้ผู้วิจัยจึงนำเสนอฟังก์ชันสูญเสียแบบใหม่ที่เรียกว่า Hybrid Loss ที่ซึ่งเป็นการผสมผสานกันระหว่าง MFE และ FL โดยการคำนวณค่าสูญเสียแต่ละตัวอย่างข้อมูลจะคำนวณด้วย FL แต่สำหรับการคำนวณค่าสูญเสียรวมจะนำวิธีการคำนวณค่าสูญเสียรวมของ MFE มาใช้ ผลการทดลองการจัดกลุ่มข้อมูลสองกลุ่มด้วยชุดข้อมูลที่หลากหลายแสดงให้เห็นว่า Hybrid Loss สามารถให้ผลการทดลองในเชิงความแม่นยำที่สูงกว่า MFE และ FL อย่างชัดเจน ซึ่งนั้นหมายความ Hybrid Loss สามารถทำงานได้อย่างดีในการจัดกลุ่มข้อมูลสองกลุ่ม ในอนาคตผู้วิจัยอาจจะทดลอง Hybrid Loss กับการจัดกลุ่มข้อมูลหลายกลุ่ม เพื่อพิสูจน์ว่าแนวคิดของ Hybrid Loss ในตอนนี้สามารถจัดการกับปัญหาความไม่สมดุลกันของข้อมูลหลายกลุ่มได้หรือไม่
    
    \clearpage
    \addcontentsline{toc}{chapter}{บรรณานุกรม}
    \bibliographystyle{IEEEtran}
    \bibliography{reference}
    
    \startappendix
    \include{appendix}
    
\end{document}